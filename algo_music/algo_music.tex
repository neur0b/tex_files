\documentclass[a4paper,12pt]{scrartcl}
\usepackage[utf8]{inputenc}
\usepackage[T1]{fontenc}
\usepackage[italian,english]{babel}
\usepackage{layaureo}
\usepackage{tgpagella}
\usepackage{hyperref}

\newcommand{\omissis}{[\dots\unkern]}

\title{\LARGE{MOSTRATE I VOSTRI SCHERMI}}


\author{} 
\subtitle{A proposito di musica, algoritmi \\ e riappropriazione rivoluzionaria}
\date{}

\begin{document}
	\maketitle
	
	
	\footnotetext{Il presente articolo è stato precedentemente pubblicato sulla rivista \emph{Altraparola}.}
	
	L'idea che l'arte rifletta le contraddizioni della società in quanto sistema complesso non è certo qualcosa di originale. Un'interpretazione dell'idea hegeliana che vede l'art come la <<manifestazione sensibile di ciò che è supremo>>, e le <<opere della bella arte come il primo anello di conciliazione tra ciò che è semplicemente esterno \omissis{} ed il puro pensiero>> \footnote{G.W.F. Hegel, \emph{Estetica}, a cura di N. Merker, Milano 1978, pp. 13-14} ci obbliga dunque a riflettere sull'arte in quanto espressione sensibile delle interpretazioni astratte del mondo, le quali -- tuttavia -- in una prospettiva immanentistica sono nel \emph{mondo stesso} e nell'atto della ragione che tenta di conoscerlo, in un continuo processo di reciproco adeguamento. Ciò significa, di conseguenza, che le categorie del sociale che facciamo rientrare nel senso comunque, e che si manifestano con grande evidenza, meritano parimenti un approfondimenti dal punto di vista estetico, per ricercarne la corrispondenza nella dimensione dell'arte. Nel capitalismo contemporaneo si assiste ad un doppio movimento contraddittorio: da un lato vi è un progressivo ed esasperante isolamento del singolo, che viene ricondotto sempre più alla sua natura di \emph{uno irrelato}, rispetto al contesto di mediazioni, dall'altro punto di vista, però, non si può negare una sempre più crescente spersonalizzazione dei processi produttivi, una sorta di scomparsa del soggetto nel processo di lavoro, in cui l'atto stesso del fare e del produrre, da sempre identificato come ciò che può dar senso all'essere individuale e sociale dell'uomo\footnote{Tesi condivisa, oltre che dalla classica visione della teoria marxista, anche dall'antropologo contemporaneo David Graeber, che definisce il \emph{non aver nulla da fare} come estremamente odioso, tanto che smentisce vigorosamente l'ipotesi secondo cui <<if humans are offered the option to be parasites, of course they'll take it>>, ribadendo che <<human beings certainly tend to ranke over what they consider excessive or degrading work;}.
	
	Da un punto di vista dell’apparire immediato – ma non per questo meno importante – si è potuto assistere, a partire soprattutto dalla seconda metà del Novecento, una progressiva ibridazione tra discorso estetico e discorso tecnologico: la razionalità della macchina, con le sue regole, i suoi movimenti precisi e, con questi, le nuove possibilità che era capace di offrire, inizia a mescolarsi con il discorso artistico, in un processo di lungo corso che, potremmo dire, affonda le sue radici nella stessa storia del pensiero occidentale. Il compositore Iannis Xenakis (1922-2001), uno dei più importanti esponenti della musica contemporanea post-seriale, che ha aperto le regole compositive alle regole matematiche delle probabilità, afferma:
	
	\begin{quotation}
		\small{<<There exists a historical parallel between European music and the successive attempts the world by reason. The music of antiquity, causal and deterministic, was already strongly influenced by the schools of Pythagoras and Plato, \omissis Scrict causality lasted until the nineteenth century when it underwent a brutal and fertile transformation as a result of statistical theories in physics>>}\footnote{I. Xenakis, \textit{Formalized Music -- Thought and mathematics in composizion,} Pendragon Press, Stuyvesant NY 1992, pp. 3-4.}.
	\end{quotation}
	
	È lo stesso pensiero occidentale, dunque, che dipanandosi a partire dalla fondamentale scoperta del principio di causalità, porta in sé la legge interna del suo sviluppo, che porterò poi, quasi \emph{naturalmente} al maccinario e, quindi, alla relazione dialettica sempre più stretta tra strumento, ragione e prodotti umani (tra i quali, ovviamente, facciamo rientrare le arti). A differenza dell'analisi condotta, ad esempio, da Adorno, qui non si vuole dare un giudizio di valore sulla crescente meccanizzazione del sapere e della produzione artistica: il principio di causalità -- radice del carattere utilitaristico della razionalità occidentale\footnote{<<L'uomo si rapporta \emph{praticamente} alla natura, come qualcosa di immediato ed esterno e con ciò stesso sensibile, che quindi anche si comporta come \emph{fine} rispetto agli oggetti della natura. La considerazione degli oggetti della natura secondo questo rapporto dà luogo al punto di vista \emph{teleologico} finito. Cfr. G.W.F. Hegel, \textit{Enciclopedia delle scienze filosofiche in compendio: Filosofia della natura}, a cura di V. Verra, UTET, Torino 2002, p. 79.} -- viene qui inteso come un \emph{dato di fatto}, alla luce del quale possiamo analizzare le forme artistiche più recenti della nostra contemporaneità. Considerarlo alla stregua di qualcosa di \emph{dato}, tuttavia, non significa affatto ricadere nel quietismo dell'analisi imparziale. Si tratta semmai di vedere quali siano i \emph{significati} insiti nelle espressioni artistiche all'interno del mondo tecnico meccanizzato, lasciando che sia il significato stesso queste opere a fornirci il corretto \emph{framework} per l'interpretazione generale, e non il contrario.
	
	Lo strumento, la macchina, che più di tutti incarna la natura logico-formale del pensiero occidentale, che manifesta in forma sensibile la natura astratta del principio di causalità e delle sue implicazioni logiche, è proprio il calcolatore, il computer. In quanto macchina \emph{calcolatrice} il computer incarna massimamente i principi logici fondamentali del nostro pensiero: la stessa natura del \emph{bit}, il fatto cioè di poter assumere soltanto due valori, \emph{zero} e \emph{uno}, ci riconduce ad una logica in cui i valori di verità siano, appunto, il \emph{vero} e il \emph{falso}. Non stupisce, dunque, che il computer possa esser considerato come la concrezione oggettiva delle categorie logiche del pensiero astratto logico-formale, sulla base del quale, ripetiamo, si è reso possibile tutto lo sviluppo tecnico dell'occidente. Ciò è massimamente interessante, dunque, quando proprio a partire dalla possibilità di calcolo del computer si inizia a ripensare la possibilità stessa dell'arte, sia in quanto processo creativo che in quanto performance. La musica d'avanguardia del Novecento ha raccolto quest'eredità tentando di liberare le costrizioni legate alla pratica musicale tradizionale, liberando il \emph{timbro}, il \emph{pitch} e la natura stessa del materiale musicale attraverso l'utilizzo di strumenti nuovi, come il nastro magnetico e il sintetizzatore, che permettevano, a partire da semplici forme d'onda, di creare timbri completamente nuovi, liberando la musica dalle costrizioni fisiche dello strumento\footnote{<<The composer, in view of the fact that he is no longer operating within a strictly ordained tonal system finds himself confronting a completely new situation. He sees himself commanding a realm of sound in which the musical material appears for the first time as a malleable continuum of every known and unknown, every conceivable and possible sound. This demands a way of thinking in new dimensions, a kind of mental adjustment to the thinking proper to the materials of electronic sound>>. Cfr. K.H. W{\"o}rner,\textit{Stockhausen: Life and Work}, University of California Press, Berkeley CA 1976, p. 123.}.
	Ma il rapporto tra musica e tecnica è ancora più profondo se considerato dal punto di vista del rapporto dellos tesso processo compositivo e le strutture logiche del pensiero. Il rapporto tra musica e strutture logiche è, ovviamente, antico. Se ne trovano tracce fin dalla stesura del \emph{Micrologus} (circa 1026) di Guido d'Arezzo, l'ideatore della notazione musicale moderna, in cui vengono enumerate le prime regole per la composizione di brani polifonici, passando per le strutture matematiche dell'\emph{Arte della fuga} di J.S. Bach e i \emph{Musikalisches W{\"u}rfelspiel} di Wolfgang Amadeus Mozart (o di qualcuno che ha deciso di utilizzare il suo nome), un vero e proprio esempio di rudimentale composizione algoritmica basata sulla combinazione di temi musicali determinati dal lancio di dadi\footnote{Cfr. M. Simoni, R.B. Dannenberg, \textit{Algorithmic composition}, The University of Mitchigan Press, Ann Arbor 2008, pp. 7-10.}. A partire poi dalla dodecafonia di Sch{\"o}nberg, passando per le estensioni del serialismo da parte di Messiaen, che estende il concetto di serie a tutti i parametri musicali, fino all'introduzione delle leggi probabilistiche nella composizione musicale da parte di Iannis Xenakis, il quale ha contribuito a fondare a parigi l'\textit{Équipe de Mathématique et d'Automatique Musicales}, il rapporto tra leggi matematiche, razionalità del pensiero scientifico e composizione musicale si è fatto così stretto fino a sovrapporsi. Uno dei risultati teoreticamente ed esteticamente più interessanti di questa commistione è rappresentato dalla \emph{ILLIAC Suite}, una composizione del 1957 "composta", appunto, dal computer ILLIAC dell'Università dell'Illinois, programmato da Lejaren Hiller e Lenoard Isaacson\footnote{\textit{Ibid.,} p. 13.}. \emph{ILLIAC suite} rappresenta probabilmente un punto di non ritorno in quanto sembra che il compositore stesso \emph{sparisca} nel processo compositivo: il compositore -- che si sovrappone alla figura del programmatore -- si limita a dare le \emph{istruzioni} al calcolatore affinché esso possa effettivamente generare una composizione musicale. L'apparente processo di spersonalizzazione dell'atto compositivo -- che sembra essere uno dei tratti più appariscenti di tutte quelle composizioni musicali denominate, appunto, \emph{algoritmiche}\footnote{<<Taken togheter \omissis the words "algorithmic music", defined by the urge to explore and/or extend musical thinking through formalized abstractions. In the process of making music as (or if you prefer, via) algorithms, we express music through formal systems of notation, taking a view of music as the higher order interplay of ideas>> Cfr. AA.VV., \textit{The Oxford Handbook of Algorithmic Music}, ed. by R.T. Dean and A. McLean, Oxford University Press, NY, 2018, pp. 5-6.} -- si ritrova anche in prodotti musicali maggiormente destinati al mercato di consumo, come ad esempio il disco \textit{Music for Airports} (1978) di Brian Eno, costruito interamente su processi generativi\footnote{<<The defining factor of generative music is that there is no way for any factor outside of the music system (software) to interact with or influence the algorithms and processes determining the end musical result. The software may include mehods of random (or nonrandom) inizialization of the sound banks, sequences, or other such musical factors, and therefore introduce the variety which defines algorithmic music, but beyond that, the listener is completely noninteractive with the system, other than that they possibly start or stop the software>>. Cfr. Y. Levtov, \textit{Algorithmic Music for Mass Consumption and Universal Production}, in \textit{Oxford Handbook of Algorithmic Music}, ed. cit., p. 628.}. A questo punto del discorso, tuttavia, sembra quasi non esserci alcun dubbio sul progressivo processo di spersonalizzazione del soggetto nel processo creativo e nell'atto compositivo: sembra quasi voler dare ragione ad Adorno. Per il filosofo francofortese questo processo di perdita dell'umano veniva espresso nei termini di un \emph{estraneazione dell'uomo nei confronti di se stesso}, completamente smarrito nei meandri dello \emph{spirito reificato}\footnote{<<Identificando in anticipo il mondo matematizzato fino in fondo con la verità, l'illuminismo si crede al sicuro dal ritorno del mito. Esso identifica il pensiero con la matematica. Essa viene, per così dire, emancipata, elevata ad istanza assoluta. \omissis L'estraneazione degli uomini dagli oggetti dominati non è il solo prezzo pagato per il dominio: con la reificazione dello spirito sono stati stregati anche i rapporti interni fra gli uomini, anche quelli di ognuno con se stesso>>. Cfr. M. Horkheimer, Th.W. Adorno, \textit{Dialettica dell'Illuminismo}, trad. R. Solmi, Einaudi, Torino, 2010, pp. 32-36.}, dominato da un pensiero matematizzante che aveva spogliato l'uomo di ogni agentività, e l'aveva fatto ripiombare nella superstizione e nel mito.
	
	Trattare la questione soltanto da questo punto di vista sarebbe tuttavia estremamente parziale e non terrebbe conto di alcuni aspetti molto interessanti in relazione ai rapporti fra la soggettività dell'uomo-artista e le sfumature logico-formali del pensiero. Le notazioni algoritmiche, che spesso e volentieri si esprimono sotto la forma di linguaggi di programmazione\footnote{Si può trattare sia di veri e propri linguaggi di programmazione dedicati esclusivamente alla composizione musicale come Csound, ChucK, SuperCollider, oppure di linguaggi di programmazione già esistenti utilizzati ai fini della composizione. Alcuni di questi sono utilizzati anche nella didattica (sia della teoria musicale che dell'informatica), come  SonicPi (cfr. \url{https://sonic-pi.net/})}, entrano in un rapporto dialettico con l'esecutore-compositore nella pratica del \emph{live coding}. Il \emph{live coding}, come si evince dallo stesso nome, è la pratica di scrivere codice finalizzato alla produzione musicale, modificandolo \emph{on-the-fly}, improvvisando (sia da soli che in collaborazione con altri performer): il processo di creazione di codice è importante tanto quanto il prodotto musicale, tanto che è prassi proiettare alle spalle del \textit{performer} quanto viene scritto ed eseguito sul laptop\footnote{<<Given the prominence of code projection and other forms of algorithmic visualization during live coding performance -- enabling audiences to form, at least to some degree, an appreciation of the music with reference to the processes by which it is being generated -- it is reasonable to assume that means of generation are integral to the aesthetic values of live coded mosical performance>>. Cfr. G.A. Wiggins and J. Forth, \textit{Computational Creativity and Live Algorithms}, in \textit{Oxford Handbook of Algorithmic Music}, ed. cit., p. 270.}. A rimarcare l'importanza del processo, che diviene predominante rispetto alla composizione in sé, è la possibilità di performance collettive e collaborative\footnote{<<\omissis in the case of collaborative live coding, code sharing is an important part of the communication, understanding each other's thoughts, and exchanging ideas. Most of the times, the collaboration in this live coding practice occurs in real time, due to its live nature and asynchronous communication being limited>>. Cfr. S.W. Lee, G. Essl, \textit{Live Writing: Asynchronous Playback of Live Coding and Writing}, p. 74 in A. McLean, T. Magnusson, K. Ng, S. Knotts, J. Armitage (ed.), published by ICSRiM, School of Music, University of Leeds, 2015.}. Riteniamo che lo stesso spirito e lo stesso afflato, nonché gli stessi nodi concettuali trasposti sul piano della performance artistica, si ritrovino nel documento più emblematico di tutto il movimento \emph{algorave}, cioè il Manifesto TOPLAP\footnote{TOPLAP è il nome dell'organizzazione informale costituitasi nel febbraio 2004 allo scopo di riunire le diverse comunità che si dedicavano alla pratica del \emph{live coding}.}. Fin dal preambolo è chara l'enfatizzazione del \emph{performer} e, di conseguenza, della sua attività produttiva di opere musicali fruibili dal pubblico: <<il vero protagonista della \emph{performance} è l'algoritmo solo in quanto modificato e plasmato dal soggetto agente: chiediamo che il codice permetta al pubblico di vedere la mente, il processo e le interpretazioni del/la livecoder, non solo il suo corpo; ci venga dato l'accesso alla mente del \emph{performer}, all'intero strumento umano>>\footnote{Cfr. la traduzione italiana del manifesto, consultabile all'indirizzo \url{https://toplapitalia.gitlab.io/manifesto.html}.}. L'accento sull'importanza del \emph{performer} (d'altronde solo in una comunità di soggetti agenti è possibile immaginare una completa apertura a pratiche collaborative) si unisce alla necessità di condividere 
	
	
	
\end{document} 
